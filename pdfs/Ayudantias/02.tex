\documentclass{ayudantia}

\title{Ayudantía 02}
\subtitle{Paradoja de los Cumpleaños}
\date{2021-04-06}
\course{IIC3253 -- Criptografía}
\faculty{Escuela de Ingeniería}
\department{Departamento de Ciencia de la Computación}
% Comment for final compile
\ifx\condition\undefined
\def\condition{2}
\fi

\ifx\condition\undefined
\immediate\write18{ pdflatex -synctex=1 -output-directory="../Enunciados" --jobname="Enunciado\jobname" "\gdef\string\condition{0} \string\input\space\jobname"} 
\immediate\write18{ pdflatex -synctex=1 -output-directory="../Soluciones" --jobname="Solucion\jobname" "\gdef\string\condition{1} \string\input\space\jobname"} 

\immediate\write18{ cd "../Enunciados" && rm *.aux *.log *.out}
\immediate\write18{ cd "../Soluciones" && rm *.aux *.log *.out}

\expandafter\stop
\fi

\ifcase\condition
\excludecomment{ans}
\or
\includecomment{ans}
\or
\includecomment{ans}
\fi

\begin{document}
\maketitle

Dado una función de \texttt{HASH} con output space \(O\) (\(n=\abs{O}\)), queremos encontrar dos cosas:
\begin{enumerate}
    \item Un \(d\in\set{N}\) ``pequeño'' tal que \(\set{P}[X=d]\geq\frac12\), donde \(\set{P}[X=d]\) es la probabilidad de que haya al menos dos colisiones después de \(d\) intentos distintos.
    \item Un \(d\in\set{N}\) ``pequeño'' tal que \(\set{E}[Y_d]\geq1\), donde \(Y_d\) es la variable aleatoria que cuenta la cantidad de colisiones  después de \(d\) intentos distintos.
\end{enumerate}

\section*{\(\set{P}[X=d]\geq\frac12\)}
Para el primero notemos que es más fácil calcular \(1-\set{P}[X=i]\) (la probabilidad de que no hayan colisiones en \(i\) intentos):
\begin{equation*}
    1-\set{P}[X=i]=\prod_{j=0}^{i-1}\paren{1-\frac{j}{n}}
\end{equation*}
Usando esto, y el hecho de que \(\mathrm{e}^x\geq1+x\), se tienen las siguientes desigualdades
\begin{align*}
    1-\set{P}[X=i]&=\prod_{j=0}^{i-1}\paren{1-\frac{j}{n}}\\
    &\leq\prod_{j=0}^{i-1}\text{exp}\paren{-\frac{j}{n}}\\
    &\leq\text{exp}\paren{\sum_{j=0}^{i-1}-\frac{j}{n}}\\
    &\leq\text{exp}\paren{\frac{-d(d-1)}{2n}}\\
    &\leq\text{exp}\paren{\frac{-(d-1)^2}{2n}}\\
\end{align*}
Por lo que si \(\text{exp}\paren{\frac{-(d-1)^2}{2n}}\leq\frac12\) se tiene que \(\set{P}[X=d]\geq\frac12\). Desarrollemos la primera expresión:
\begin{align*}
    \text{exp}\paren{\frac{-(d-1)^2}{2n}}&\leq\frac12\\
    \frac{-(d-1)^2}{2n}&\leq\log\frac12\\
    (d-1)^2&\geq-2n\log\frac12\\
    (d-1)^2&\geq2n\log2\\
    d-1&\geq\sqrt{2n\log2}\\
    d&\geq\sqrt{2n\log2}+1\\
\end{align*}
Con lo que tomando \(d=\ceil{\sqrt{2n\log2}+1}\) se tiene la desigualdad.
\section*{\(\set{E}[Y_d]\geq1\)}
Para calcular el segundo, notemos que \(Y_d=\sum_{i=1}^d\sum_{j=i+1}^dX_{ij}\) donde \(\set{P}[X_{ij}=1]=\frac1n\) y \(\set{P}[X_{ij}=k]=0\) para \(k\neq1\). Por lo que \(\set{E}[Y_d]=\sum_{i=1}^d\sum_{j=i+1}^d\set{E}[X_{ij}]=\frac{d(d-1)}{2n}\geq\frac{(d-1)^2}{2n}\), por lo que tomando \(d=\ceil{\sqrt{2n}+1}\) es suficiente.
\end{document}