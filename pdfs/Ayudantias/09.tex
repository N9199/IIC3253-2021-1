\documentclass{ayudantia}

\title{Ayudantía 09}
\subtitle{Euler-Fermat y Teorema Fundamental de la Aritmética}
\date{2021-05-26}
\course{IIC3253 -- Criptografía}
\faculty{Escuela de Ingeniería}
\department{Departamento de Ciencias de la Computación}
% Comment for final compile
\ifx\condition\undefined
\def\condition{2}
\fi

\ifx\condition\undefined
\immediate\write18{ pdflatex -synctex=1 -output-directory="../Enunciados" --jobname="Enunciado\jobname" "\gdef\string\condition{0} \string\input\space\jobname"} 
%\immediate\write18{ pdflatex -synctex=1 -output-directory="../Soluciones" --jobname="Solucion\jobname" "\gdef\string\condition{1} \string\input\space\jobname"} 

\immediate\write18{ cd "../Enunciados" && rm *.aux *.log *.out}
\immediate\write18{ cd "../Soluciones" && rm *.aux *.log *.out}

\expandafter\stop
\fi

\ifcase\condition
\excludecomment{ans}
\or
\includecomment{ans}
\or
\includecomment{ans}
\fi

\begin{document}
\maketitle
\section{Euler-Fermat}
\subsection{Recordatorio}
\begin{thm}[Inverso Modular]
    Dado \(a\in\set{Z}\), existe \(a^{-1}\in\set{Z}\) tal que \(a\cdot a^{-1}\equiv1\mod n\) si y solo si \(\gcd(a,n)=1\).
\end{thm}
\begin{defn}[\(\set{Z}_n^*\)]
    Se define \(\set{Z}_n^*\) como el conjunto \(\{a\in\set{Z}\mid 1\leq a<n\wedge\gcd(a,n)=1\}\).
\end{defn}
\begin{defn}[Función \(\varphi\)]
    La función \(\varphi:\set{N}\rightarrow\set{N}\) se define como
    \begin{equation*}
        \varphi(n)=\abs{\set{Z}_n^*}.
    \end{equation*}
\end{defn}
\subsection{Teorema Euler-Fermat}
\begin{thm}[Euler-Fermat]
    Dado \(a\in\set{Z}\) tal que \(\gcd(a,n)=1\), se tiene que
    \begin{equation*}
        a^{\varphi(n)}\equiv1\mod n
    \end{equation*}
\end{thm}
\begin{proof}
    Se toma la función
    \begin{align*}
        f_a:&\set{Z}_n^*\rightarrow\set{Z}_n^*\\
        &x\mapsto (ax)\mod n
    \end{align*}
    y se nota que es una biyección, por lo que se tiene la siguiente congruencia:
    \begin{align*}
        x_1\cdot\ldots\cdot x_{\varphi(n)}&\equiv (ax_1)\cdot\ldots\cdot (ax_{\varphi(n)})\mod n\\
        &\equiv a^{\varphi(n)}x_1\cdot\ldots\cdot x_{\varphi(n)}\mod n
    \end{align*}
    Multiplicando por los inversos de cada \(x_i\) a cada lado se llega a lo pedido.
\end{proof}
\begin{prob}
    Demuestre que \(f_a\) es una biyección. (\textit{Hint: demuestre que \(f_{a^{-1}}\) es su inversa.})
\end{prob}
\begin{thm}
    Dado \(p,q\) primos distintos se tiene que \(\varphi(p)=p-1\) y \(\varphi(pq)=(p-1)(q-1)\).
\end{thm}
\begin{proof}
    El primer resultado es inmediato ya que para todo \(1\leq a<p\) se tiene que \(\gcd(a,p)=1\) por definición de primo. Para el segundo resultado, se nota que se pueden contar los \(a\) tal que \(\gcd(a,pq)>1\), estos son los números que son divisibles por \(p\) o por \(q\) que son menores a \(pq\), como el menor entero positivo divisible  por \(p\) y por \(q\) es \(pq\) se tiene que solo hay que contar enteros divisibles por \(p\) y después enteros divisibles por \(q\). Se nota que los enteros divisible por \(p\) son \(p,2p,\ldots, (q-1)p\), que son \(q-1\), y similarmente hay \(p-1\) divisibles por \(q\). Por lo que \(\varphi(pq)=(pq-1)-(q-1)-(p-1)=pq-p-q+1=(p-1)(q-1)\).
\end{proof}
\begin{prob}
    Demuestre que \(\varphi(p^n)=p^{n-1}(p-1)\) para \(p\) primo.(\textit{Hint: cuente los que no son coprimos.})
\end{prob}
\section{Teorema Fundamental de la Aritmética}
\begin{thm}
    Dado \(a,b,c\in\set{Z}\) tales que \(\gcd(a,b)=1\) y \(a\mid bc\), entonces \(a\mid c\).
\end{thm}
\begin{proof}
    Como \(a\mid bc\) se tiene que \(ak=bc\) para algún \(k\in\set{Z}\). Usando Bezout se tiene que \(ax+by=1\) para algunos \(x,y\in\set{Z}\), multiplicando todo por \(c\) se tiene que \(acx+bcy=c\) y reemplazando se tiene \(a(cx+ky)=c\), con lo que se concluye que \(a\mid c\).
\end{proof}
\begin{lem}
    Dado \(p\) primo y \(a,b\in\set{Z}\), se tiene que si \(p\mid ab\) entonces \(p\mid a\) o \(p\mid b\).
\end{lem}
\begin{proof}
    Si \(p\mid a\), se tiene el resultado. En otro caso, se tiene que \(\gcd(a,p)=1\), por lo que por el teorema anterior se tiene que \(p\mid b\).
\end{proof}
\begin{thm}[Teorema Fundamental de la Aritmética]
    Todo entero descomposición en producto de primos, y es única salvo reordenamientos. 
\end{thm}
\begin{proof}
    Para la existencia se usará inducción fuerte. El paso base es \(n=2\), se sabe que \(2\) es primo, por lo que se tiene el caso base. Dado \(n\) si es primo, se tiene lo pedido, si no, existen \(1<a,b<n\) tales que \(n=ab\), y por hipótesis inductiva se tiene que \(a=p_1\cdot\ldots\cdot p_k\) y \(b=q_1\cdot\ldots\cdot q_l\), por lo que \(ab=p_1\cdot\ldots\cdot p_k\cdot q_1\cdot\ldots\cdot q_l\), con lo que se tiene lo pedido.

    Para la unicidad dado \(n\) se toman dos descomposiciones primas, \(n=p_1\cdot\ldots\cdot p_k=q_1\cdot\ldots\cdot q_l\). Dado \(p_1\), como \(p_1\mid n\) por el lema anterior se tiene que \(p_1\mid q_i\) para algún \(i\), y por definición de primo se tiene que \(p_1=q_i\), dividiendo ambos lados por \(p_1\) se tiene el mismo caso, pero con menos primos, por lo tanto se tiene lo pedido.
\end{proof}
\begin{remark}
    Dado que los primos tienen un orden natural, hay una descomposición canónica de cada entero \(n\), donde \(n=p_1^{\alpha_1}\cdot\ldots\cdot p_k^{\alpha_k}\), \(\alpha_i\geq 0\) y \(p_i\) es el \(i\)-ésimo primo..
\end{remark}
\subsection{Aplicaciones y Problemas}
\begin{prob}
    Demuestre que dado \(a,b\in\set{Z}\), donde \(a=p_1^{\alpha_1}\cdot\ldots\cdot p_k^{\alpha_k}\) y \(b=p_1^{\beta}\cdot\ldots\cdot p_k^{\beta}\), se tiene que \(\gcd(a,b)=p_1^{\min(\alpha_1,\beta_1)}\cdot\ldots\cdot p_k^{\min(\alpha_k,\beta_k)}\).
\end{prob}
\begin{ejm}
    Se tienen \(a=24\) y \(b=60\), se nota que \(\gcd(a,b)=12=2^2\cdot 3\), y que \(a=2^3\cdot 3\) y \(b=2^2\cdot3\cdot5\).
\end{ejm}
\begin{prob}
    Dados \(a,b\in\set{Z}\) tales que \(\gcd(a,b)=1\) y \(ax=by\) demuestre que \(a\mid y\) y \(b\mid x\) usando TFA.
\end{prob}
\begin{proof}
    Demostración sin TFA: Se nota que \(a\mid by\) por lo que por teorema demostrado anteriormente se tiene que \(a\mid y\), similarmente se tiene que \(b\mid x\).
\end{proof}
\begin{prob}
    Demuestre usando TFA que si \(a^2\mid b^2\) entonces \(a\mid b\).
\end{prob}


\begin{prob}
    Demuestre lo anterior sin TFA.
\end{prob}


\begin{prob}[Bonus]
    Demuestre que existe una función \(f\) de \(\set{N}\) a las secuencias de naturales con finitos términos no ceros\footnote{De forma más precisa, al conjunto \(\{\{a_n\}_{n\in\set{N}}\mid \forall n\in\set{N}\, a_n\in\set{N}\text{ y para finitos \(n\) se tiene que }a_n\neq0\}\).} tal que \(f(a\cdot b)=f(a)+f(b)\), si y solo si se tiene TFA y existen infinitos primos.
\end{prob}

\end{document}