\documentclass{ayudantia}

\title{Ayudantía 07}
\subtitle{Generadores Pseudo-Aleatorios y Funciones One-Way}
\date{2021-05-05}
\course{IIC3253 -- Criptografía}
\faculty{Escuela de Ingeniería}
\department{Departamento de Ciencias de la Computación}
% Comment for final compile
\ifx\condition\undefined
\def\condition{2}
\fi

\ifx\condition\undefined
\immediate\write18{ pdflatex -synctex=1 -output-directory="../Enunciados" --jobname="Enunciado\jobname" "\gdef\string\condition{0} \string\input\space\jobname"} 
\immediate\write18{ pdflatex -synctex=1 -output-directory="../Soluciones" --jobname="Solucion\jobname" "\gdef\string\condition{1} \string\input\space\jobname"} 

\immediate\write18{ cd "../Enunciados" && rm *.aux *.log *.out}
\immediate\write18{ cd "../Soluciones" && rm *.aux *.log *.out}

\expandafter\stop
\fi

\ifcase\condition
\excludecomment{ans}
\or
\includecomment{ans}
\or
\includecomment{ans}
\fi

\begin{document}
\maketitle
\section{Generadores Pseudo-Aleatorios}
\begin{defn}[Generador Pseudo-Aleatorio (PRG)]
    Sea \(\ell(\cdot)\) un polinomio y sea \(G\) un algoritmo determinista en \texttt{P-TIME} tal que para toda entrada \(s\in\{0,1\}^n\), \(G\) devuelve un string de largo \(\ell(n)\). Se dice que \(G\) es PRG si:
    \begin{enumerate}
        \item (Expansión): \(\forall n\,\ell(n)>n\).
        \item (Pseudo-Aleatoriedad): Para todo distinguidor en \texttt{RP-TIME} \(D\) se tiene que existe una función despreciable \(f\) tal que:
        \begin{equation*}
            \abs{\underset{r\leftarrow\{0,1\}^{\ell(n)}}{P}[D(r)=1]-\underset{s\leftarrow\{0,1\}^n}{P}[D(G(s))=1]}\leq f(n)
        \end{equation*}
    \end{enumerate}
\end{defn}
\begin{remark}
    Se dice que \(\ell\) es ``el factor de expansión de \(G\)''.
\end{remark}
\begin{ejm}
    Un \textbf{\underline{mal}} ejemplo es \(G(k)=(a_{\abs{k}}\cdot k+b_{\abs{k}})\mod c_{\abs{k}}\), donde \(k\in\{0,1\}^*\) y es interpretado como un número en binario, y donde \(G(k)\) se reexpresa como un string binario.

    Para ver que es un mal ejemplo, primer notemos que \(\ell(n)=\abs{c_n}\), por lo que si \(\abs{c_n}>n\) se tiene \(1\), y esto es trivialmente cierto eligiendo \(c_n\) suficientemente grandes. Para \(2\), notemos que el objetivo del distinguidor es saber si la congruencia \(r\equiv a_n\cdot k+b_n\mod c_n\) tiene solución. Recordamos que existe un algoritmo en \texttt{P-TIME} que revisa si tiene solución, y si existe devuelve una solución, por lo que el distinguidor puede correr este algoritmo y retornar \(1\) cuando hay solución. Con lo anterior se tiene que 
    \begin{align*}
        \underset{r\leftarrow\{0,1\}^{\ell(n)}}{P}[D(r)=1]=\frac{S_n}{2^{\ell(n)}}\text{ y }\underset{s\leftarrow\{0,1\}^n}{P}[D(G(s))=1]=1,
    \end{align*}
    donde \(S_n\) es la cantidad de posibles congruencias que tienen soluciones \(k\) tales que \(\abs{k}=n\). Se sabe que \(S_n\leq 2^n\), por lo que se tiene
    \begin{align*}
        \underset{r\leftarrow\{0,1\}^{\ell(n)}}{P}[D(r)=1]&=\frac{S_n}{2^{\ell(n)}}\\
        &\leq\frac{2^n}{2^{\ell(n)}}\\
        &\leq\frac1{2^{\ell(n)-n}}\\
        &\leq\frac12\\
    \end{align*}
    y así se llega a que \(\abs{\underset{r\leftarrow\{0,1\}^{\ell(n)}}{P}[D(r)=1]-\underset{s\leftarrow\{0,1\}^n}{P}[D(G(s))=1]}\geq\frac12\), por lo que no cumple 2.
\end{ejm}
\end{document}