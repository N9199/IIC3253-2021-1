\documentclass{ayudantia}

\title{Ayudantía 08}
\subtitle{}
\date{2021-05-19}
\course{IIC3253 -- Criptografía}
\faculty{Escuela de Ingeniería}
\department{Departamento de Ciencias de la Computación}
% Comment for final compile
\ifx\condition\undefined
\def\condition{2}
\fi

\ifx\condition\undefined
\immediate\write18{ pdflatex -synctex=1 -output-directory="../Enunciados" --jobname="Enunciado\jobname" "\gdef\string\condition{0} \string\input\space\jobname"} 
\immediate\write18{ pdflatex -synctex=1 -output-directory="../Soluciones" --jobname="Solucion\jobname" "\gdef\string\condition{1} \string\input\space\jobname"} 

\immediate\write18{ cd "../Enunciados" && rm *.aux *.log *.out}
\immediate\write18{ cd "../Soluciones" && rm *.aux *.log *.out}

\expandafter\stop
\fi

\ifcase\condition
\excludecomment{ans}
\or
\includecomment{ans}
\or
\includecomment{ans}
\fi

\begin{document}
\maketitle
\section{Recordatorio}
\begin{defn}[Generador Pseudo-Aleatorio (PRG)]
    Sea \(\ell(\cdot)\) un polinomio y sea \(G\) un algoritmo determinista en \texttt{P-TIME} tal que para toda entrada \(s\in\{0,1\}^n\), \(G\) devuelve un string de largo \(\ell(n)\). Se dice que \(G\) es PRG si:
    \begin{enumerate}
        \item (Expansión): \(\forall n\,\ell(n)>n\).
        \item (Pseudo-Aleatoriedad): Para todo distinguidor en \texttt{RP-TIME} \(D\) se tiene que existe una función despreciable \(f\) tal que:
        \begin{equation*}
            \abs{\underset{r\leftarrow\{0,1\}^{\ell(n)}}{P}[D(r)=1]-\underset{s\leftarrow\{0,1\}^n}{P}[D(G(s))=1]}\leq f(n)
        \end{equation*}
    \end{enumerate}
\end{defn}

\begin{defn}[Esquema de Cifrado Basado en PRG]
    Sea \(G\) un PRG con factor de expansión \(\ell\). Se define el esquema de cifrado de llave privada para mensajes de largo \(\ell(n)\) a continuación:
    \begin{itemize}
        \item \texttt{Gen}: dado \(1^n\), elige \(k\leftarrow\{0,1\}^n\) de forma uniformemente aleatoria.
        \item \texttt{Enc}: dado una llave \(k\in\{0,1\}^n\) y un mensaje \(m\in\{0,1\}^{\ell(n)}\), retorna el texto cifrado
        \begin{equation*}
            c:=G(k)\oplus m.
        \end{equation*}
        \item \texttt{Dec}: dado una llave \(k\in\{0,1\}^n\) y un texto cifrado \(c\in\{0,1\}^{\ell(n)}\), retorna el mensaje
        \begin{equation*}
            m:=G(k)\oplus c.
        \end{equation*}
    \end{itemize}
\end{defn}

\begin{defn}[Cifrado Indistinguible ante Ataque de Texto Cifrado]
    Dado \texttt{Enc} se define el siguiente juego:
    \begin{enumerate}
        \item El Adversario elige \(m_0,m_1\in\{0,1\}^{\ell(n)}\).
        \item El Verificador elige \(b\in\{0,1\}\) y le devuelve a \(\texttt{Enc}(k,m_b)\) al Adversario.
        \item El Adversario indica si \(b=0\) o \(b=1\).
    \end{enumerate}
    Si \(P(A)\leq\frac12+f(n)\), donde \(A\) es el evento que el Adversario gane y \(f\) es una función despreciable se dice que \texttt{Enc} es un cifrado indistinguible ante un ataque de texto cifrado.
\end{defn}

\begin{thm}
    Dado un PRG, \(G\), se tiene que el esquema basado en \(G\), \(\paren{\texttt{Gen},\texttt{Enc},\texttt{Dec}}\), tiene cifrado indistinguible ante un ataque de texto cifrado,
\end{thm}
\begin{proof}
    Se usará contrapositiva para demostrar el teorema. Sea \(A\) el algoritmo en \texttt{RP-TIME} tal que \(P(A_{wins})>\frac12+f(n)\) para toda función despreciable \(f\) y donde \(A_{wins}\) es el evento donde el algoritmo \(A\) gana el juego de la definición 1.3. Se define \(\varepsilon(n):=P(A_{wins})-\frac12\), y se nota que \(\varepsilon\) no es una función despreciable\footnote{\(P(A_{wins})\leq\varepsilon(n)=\frac12\).}. Con lo anterior se construye un algoritmo distinguidor D:
\begin{center}
    \fbox{\begin{minipage}{15em}

    \end{minipage}}
\end{center}
\end{proof}
\end{document}